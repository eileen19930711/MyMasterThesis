% !TeX spellcheck = de_DE

\chapter{Conclusion and Future Work}
\label{chap:conclusion}

This thesis exploits the use of linear regression in image space for 3D lane markings reconstruction. The presented approach showed that despite of the poorly textured neighborhoods, the lane markings can be reconstructed with ???centimeter accuracy??? using well configured aerial images. Instead of using explicit line matching algorithms, the approach requires initial approximate 3D lines and is highly dependent on quality of known image orientations. Nevertheless, it is robust to partial occlusion and is applicable on the (quasi-)infinite lines. %in lowly textured neighberings cases. ???
The approach is developed for both dashed and continuous lane lines.???




Some conclusions are given:

%\cref{sec:Geometry}
With the extracted line in the form of sets of points with sub-pixel accuracy and EOIO accuracy???, linear regression can be applied on reconstructing the 3D line segment position.

image distortion vs straight line approximation


image configuration plays an important roll in 3D reconstruction. Being covered from more views





For SGM generated DSM, lane markings can be exploited to provide information for DSM refinement. 

lane markings are one of the most valuable features on road surface reconstruction to provide 

 refinement based on triangulation is necessary.



Influence of 


advantage

The approach can be used on lane markings reconstruction

not relying on textures or appearance-based matching
but purely based on imaging geometry.

drawback

unknown point-to-point relationship in image space




\section*{Future Work}
\label{chap:futurework}




