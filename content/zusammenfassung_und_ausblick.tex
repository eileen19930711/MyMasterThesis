% !TeX spellcheck = de_DE

\chapter{Conclusion and Future Work}
\label{chap:conclusion}

This thesis exploits the use of linear regression in image space with combination of collinearity condition for 3D lane markings reconstruction. Without utilization of the neighboring textures, the approach requires initial approximate 3D lines and is highly dependent on quality of pre-known image orientations. Nevertheless, it is robust to partial occlusion and is applicable on the (quasi-)infinite lines as well as in cases of lowly textured neighborings. 
%The approach is developed for both dashed and continuous lane lines.???

From simulated experiments in \cref{sec:simulation}, some conclusions are given:
\begin{itemize}
	\item Given approximations with 2 meters bias in Z-direction and sub-pixel random noises in the image coordinates of the detected/measured 2D lines, the proposed approach correctly refined the 3D positions of line segments.
	
	\item With the same line orientations in 3D space, the configuration strength increases with the increase of amount of covering images from different views.
	
	\item The reconstructed line segments have higher precision in vertical direction than in horizontal direction.
\end{itemize}


From experimental results of true data in \cref{sec:simulation}, some conclusions are given:
\begin{itemize}
	\item The initial approximations, i.e. the DSM profile, are significantly far away from the reconstructed line segments. This indicates the necessity of reconstrucion based on detected 2D lines in image space and the viewing geometry, instead of simply reducing DSM noise by applying mean filter or such.
	
	\item The LS-estimated quality of the extracted lines as well the image orientations is better than 1 pixel. %especially the posterior standard deviation contains the errors of the imperfect image orientation parameters.
	
	\item Different line orientations in 3D space have different configuration strength despite of the same covering images amount from different views.
	
	\item The configuration defect in image space %which is mentioned in \cref{subsec:LSadj} 
	barely happens when the functional models between image coordinates $x$ and $y$ are chosen properly for linear regression ---accordingly to the characters of detected lane lines in image space. %In the cases of flying along the motorways, the detected lane lines lie mainly in column direction on images. In other words, the defection in image space can be avoided by .
	
	\item  The configuration defect in object space %which is mentioned in \cref{subsec:LSadj} 
	may happen in some of the stereo views in the same strip with the adopted flight configuration in this work.% of flying along the motorways, the detected lane lines lie mainly in flight direction on images.
	This indirectly indicates that under this aerial images configuration increasing forward overlapping rate does not really increase the configuration strength for 3D lane markings reconstruction.
	
\end{itemize}



...

using precisely detected lines with sub-pixel precision as well the bundle adjusted image orientations, the lane markings can be reconstructed with ???centimeter accuracy??? using well configured aerial images.


%\cref{sec:Geometry}
With the extracted line in the form of sets of points with sub-pixel accuracy and EOIO accuracy???, linear regression can be applied on reconstructing the 3D line segment position.

image distortion vs straight line approximation


image configuration plays an important roll in 3D reconstruction. Being covered from more views



For SGM generated DSM, lane markings can be exploited to provide information for DSM refinement. 

lane markings are one of the most valuable features on road surface reconstruction to provide 

 refinement based on triangulation is necessary.



Influence of 


advantage

The approach can be used on lane markings reconstruction

not relying on textures or appearance-based matching
but purely based on imaging geometry.

drawback

unknown point-to-point relationship in image space




\section*{Future Work}
\label{chap:futurework}

Future improvements could be achieved by:

increasing the oblique angle


