% !TeX spellcheck = de_DE

\chapter{Methodology}
\label{chap:k2}

The following sections describe the principles of 3D lane marking reconstruction method proposed in this thesis. 
% with respect to their location in the processing chain (Fig.1.).

Section~\ref{sec:LineExtraction} describes how lane markings are extracted.
Section~\ref{sec:Geometry} introduces the geometric properties of aerial images and their mathematical models, including collinearity equation and lens distortion correction. 
Section~\ref{sec:LineFitting} describes the principle of line fitting as well as line equation in two-point form.
Modeled through non-linear least-squares adjustment, section~\ref{sec:2.4} elaborates the usage of line fitting on 3D lane marking reconstruction, which is the key approach to solve the non-textured-neighboring quasi-infinite line reconstruction problem.


Fig. 1. Processing chain




%%%%%%%%%%%%%%%%%%%%%%%%%%%%%%%%%%%%%%%%%%%%%%%%%%%%%%%%
\section{Lane markings Extraction}
\label{sec:LineExtraction}
[This is done with Halcon11/C...]
with subpixel precision
Each extracted line is rasterized as a set of points in image coordinate (with … precision).
line (2D) -> a set of points (2D) , in image space
% http://www.mvtec.com/doc/halcon/11/en/lines_gauss.html






%%%%%%%%%%%%%%%%%%%%%%%%%%%%%%%%%%%%%%%%%%%%%%%%%%%%%%%%
\section{Geometric Properties of Aerial Photographs}
\label{sec:Geometry}

This section describes the geometric model of the projection of 3D points into the image generated by a real camera. We first restrict the discussion in subsection~\ref{subsec:Collinearity} to central perspective projection where the collinearity equation originate from. We then model deviations from this model, addressing real cameras with imperfect lenses, in subsection~\ref{subsec:LensDistortion}.

\subsection{Collinearity Equations}
\label{subsec:Collinearity}
We assume frame photography, i.e. photographs exposed in one instant, and assume central projection model, i.e. cameras who have a single viewpoint and a planar sensor and being straight line-preserving. Collinearity indicates the condition that the image point (on the sensor plate of the camera), the observed point (on the object) and the projection center of the camera were aligned at the moment the picture was taken. Every measured point leads to two collinearity equations, describing transformations from object space to image coordinates:
\begin{equation} %\label{eq:2.3}
\begin{split}
x - x_0 = -c \dfrac {R_{11}(X-X_0) + R_{21}(Y-Y_0) + R_{31}(Z-Z_0)} {R_{13}(X-X_0) + R_{23}(Y-Y_0) + R_{33}(Z-Z_0)} \\
y - y_0 = -c \dfrac {R_{12}(X-X_0) + R_{22}(Y-Y_0) + R_{32}(Z-Z_0)} {R_{13}(X-X_0) + R_{23}(Y-Y_0) + R_{33}(Z-Z_0)}
\end{split}
\end{equation}
where\newline
$(x, y)$: image coordinates of the point \newline
$(x_0, y_0)$: image coordinates of principal point \newline
$c$: principal distance \newline
$(X, Y, Z)$: object coordinates of the point \newline
$(X_0, Y_0, Z_0)$: object coordinates of projection center \newline
$R_{11},...,R_{33}$: elements of the rotation matrix R (orthogonal 3$\times$3-matrix from object space to image space, with 3 independent angles $\omega$, $\phi$ and $\kappa$)

The exterior orientations: $X_0$, $Y_0$, $Z_0$, $\omega$, $\phi$, $\kappa$ as well as the interior orientations: $x_0$, $y_0$, $c$ are the bundle-adjusted, known elements in this work.
%The image coordinates $x,y$ are the measurements.

\subsection{Lens Distortion Correction}
\label{subsec:LensDistortion}

As real cameras generally only approximate the perspective camera model, lens distortion correction can be additionally included in the collinearity model, attempting to correct the position of image features such that they obey the perspective model with sufficient accuracy.[W. Förstner et al. 2016] 

Varies models for lens distortion ...

A subset of a physical distortion model[] is chosen for...
Assuming $x$ and $y$ to be the distorted image coordinates, the corrections $\Delta x$ and $\Delta y$ are then calculated by the following equations
\begin{equation} %\label{eq:2.2}
\begin{split}
\Delta x &= x_p + A_1x_*(r^2-R_0^2) + A_2x_*(r^4-R_0^4) + B_1(r^2+2x_*^2) + B_22x_*y+C_2y \\
\Delta y &= y_p + A_1y  (r^2-R_0^2) + A_2y  (r^4-R_0^4) + B_1(r^2+2y^2)   + B_22x_*y
\end{split}
\end{equation}
with $r=\sqrt{x_*^2+y^2}$ and $x_*=\dfrac{x}{C_1}$. The undistorted image coordinates $x\prime$ and $y\prime$ are then calculated by $x\prime=x+\Delta x$ and $y\prime=y+\Delta y$. [F. Kurz et al. 2012] 

%This allows us to reconstruct the viewing rays from the corresponding set of image points.






%%%%%%%%%%%%%%%%%%%%%%%%%%%%%%%%%%%%%%%%%%%%%%%%%%%%%%%%
\section{Line Fitting}
\label{sec:LineFitting}

Line fitting is the process of constructing a infinite straight line that has the best fit to a 2D dataset. One of the approaches is linear regression which attempts to model the relationship between two variables by fitting a linear equation to observed data. % [www.stat.yale.edu/Courses/1997-98/101/linreg.htmf]
Additional least-squares (LS) models are commonly used for regression by means of minimization of sum of squared residuals.

In the case of simple linear regression as presented in subsection~\ref{subsec:LinearRegression}, the independent variable $x$ is error free, inconsistencies are only for the dependent variable $y$. Geometrically it means that the vertical distances from observed data to the fitted line is minimized. To minimize the sum of squared perpendicular distances from the data points to the regression line, a LS mixed model is derived in subsection~\ref{subsec:MixedModel} to perform orthogonal regression.

For a later combination with collinearity equations from subsection~\ref{subsec:Collinearity}, we aim to fit the line equation in two-point form to the extracted lines from section~\ref{sec:LineExtraction} (in forms of sets of points in image space). For such purpose, a non-linear LS adjustment model is derived in subsection~\ref{subsec:NonLinear}. 


\subsection{Linear Regression Model}
\label{subsec:LinearRegression}

Given a data set $\{x_i,y_i\}^n_{i=1}$ of $n$ points on a 2D plane, a linear regression model assumes that the relationship between the dependent variable $y_i$ and the regressors $x_i$ is linear. This relationship is modeled through a error variable $e_{y_i}$ -an unobserved random variable that adds noise to the linear relationship between the dependent variable and regressors.
Thus the model takes the form:
\begin{equation} %\label{eq:2.1}
y_i - e_{y_i} = a_0 + a_1x_i
\end{equation}
%https://en.wikipedia.org/wiki/Linear_regression#Assumptions


\subsection{Mixed Model (Gauss-Helmert Model)}
\label{subsec:MixedModel}

In the case with inconsistencies $e_{x_i}$ and $e_{y_i}$ in both observations $x_i$ and $y_i$, A-model with pseudo observation equations can be formulated:
\begin{equation} %\label{eq:2.2}
y_i - e_{y_i} = a_0 + a_1(x_i-e_{x_i}) = a_0 + a_1\bar{x_i}
\end{equation}
\begin{equation} %\label{eq:2.2}
x_i-e_{x_i} = \bar{x_i}
\end{equation}
In the A-model, every observation is either a linear or a non-linear function of all unknown quantities.In contrast, in the B-model no unknown parameter exist and there are linear or non-linear relationships between the observations [F. Krumm]. 
% ~\ref{eq.???}
As eq.?2.4? describes the relationship between observation and unknowns and eq.?2.5? the relationship between observations, the whole set is known as the mixed model or Gauss-Helmert model.




\subsection{Non-linear Least-Squares Adjustment Model}
\label{subsec:NonLinear}

Line Equation in two-point form:
\begin{equation} %\label{eq:2.4}
y-y_1 = \dfrac{(y_2-y_1)}{(x_2-x_1)}\times(x-x_1)
\end{equation}
where two points $(x_1,y_1)$ and $(x_2,y_2)$ define the infinite line with $x_2\neq x_1$ and $(x,y)$ is any point on the line.

Let the unknown image coordinates of the start- and endpoints of a line be $(x_1,y_1)$ and $(x_2,y_2)$, and the observations $\{x_i,y_i\}^n_{i=1}$. Rewriting equation ?2.4? forms the observation equation:
\begin{equation} %\label{eq:2.3}
y_i = \dfrac{(y_2-y_1)}{(x_2-x_1)}\times x_i+(y_1-\dfrac{(y_2-y_1)}{(x_2-x_1)}\times x_1) + e_{y_i}
\end{equation}

The observations $y_i$ are non-linear functions of all unknown quantities.

% % %
Taylor expansion...(local linear approximation)
the linear equation can also be expanded






%%%%%%%%%%%%%%%%%%%%%%%%%%%%%%%%%%%%%%%%%%%%%%%%%%%%%%%%
\section{Line Projection on DSM}
\label{sec:}

Given image coordinates $(x,y)$ of a point and the (bundle-adjusted) image orientations, there is still one degree of freedom in collinearity equation system Equa. on solving object coordinates $(X,Y,Z)$. Combined with the usage of DSM, who provides an initial height information given a position $(X,Y)$, the corresponding object coordinates can be solved iteratively.

With the following work flow:

pseudo code:

%x,y = a set of points (detected line);		// unit: [pixel], in image coordinates
%Z_ini = 500;						// unit: [meter], in world coordinates
%while( ( Z_new – Z_ini ) < convergentthreshold )
%	(X,Y) = camera.img2geo(x,y,Z_ini);	// X,Y in world coordinates
%	Z_new = DSM.Get_height(X,Y);		// Z_new in world coordinates
%end
%return (X,Y,Z_new);

Considering that X,Y have continuous numerical values whereas the DSM is raster (discrete), bilinear interpolation is adopted in this work.

% % %
to provide the initial approximation for non-linear LS adj





%%%%%%%%%%%%%%%%%%%%%%%%%%%%%%%%%%%%%%%%%%%%%%%%%%%%%%%%
\section{Least-Squares Adjustment Model with constraints}
\label{sec:}


\subsection{constraints}
\label{subsec:}

% % %
C matrix
D matrix

\subsection{LS Adjustment with Constraints}
\label{sec:}
Line fitting in image space with unknowns in object coordinates


% % %
derive matrices
extended normal equation matrix N*



%%%%%%%%%%%%%%%%%%%%%%%%%%%%%%%%%%%%%%%%%%%%%%%%%%%%%%%%
\section{Line Grouping}
\label{sec:}





%\begin{equation}\label{eq:test}
%a = b + c.
%\end{equation}

%\begin{equation}
%a = b + c. \tag{\ref{eq:test} revisited}
%\end{equation}



