% !TeX spellcheck = de_DE

\chapter{Experimental Results and Evaluations}
\label{chap:k3}

evaluation procedure
The distance between the LIDAR points and the generated DSM are computed and evaluated statistically
To evaluate the influence of the over-counting correction and MGM
cost aggregation independently, two evaluations were performed






%%%%%%%%%%%%%%%%%%%%%%%%%%%%%%%%%%%%%%%%%%%%%%%%%%%%%%%%
\section{The Correctness of the LS adjustment model}

This section aims to verify the correctness of LS model

\subsection{Simulation data without noise}
initial value changed

\subsection{Simulation data with random noise}




%%%%%%%%%%%%%%%%%%%%%%%%%%%%%%%%%%%%%%%%%%%%%%%%%%%%%%%%
\section{True Data: Short Dashed Lane markings}



%%%%%%%%%%%%%%%%%%%%%%%%%%%%%%%%%%%%%%%%%%%%%%%%%%%%%%%%
\section{True Data: Long Lane markings}


%%%%%%%%%%%%%%%%%%%%%%%%%%%%%%%%%%%%%%%%%%%%%%%%%%%%%%%%
3.1  Reference Data
The primary reference dataset used in this paper is a 3D point
cloud acquired by airborne laser scanning with a density of ap-
proximately 0.5 points per square meter.  The laser point cloud
data is georeferenced in UTM Zone 31 North, ETRS89 and con-
tains orthometric heights with respect to EGM 2008.  The ortho-
metric  heights  were  converted  to  ellipsoidal  heights  by  simply
adding the undulations from EGM 2008.  Only the first pulse re-
turns is used in this study, as the DSM produced by image match-
ing corresponds to the visible surface.  The LIDAR data for the
Terrassa and Vacarisses test areas was acquired on 26th and 27th
November 2007. The LaMola LIDAR data was acquired on 26th
November 2007 and 4th May 2008



%%%%%%%%%%%%%%%%%%%%%%%%%%%%%%%%%%%%%%%%%%%%%%%%%%%%%%%%
4.3 Internal Factors

4.3.1 Correctness of

4.3.2 Verification of

4.3.3 Precision of


%%%%%%%%%%%%%%%%%%%%%%%%%%%%%%%%%%%%%%%%%%%%%%%%%%%%%%%%
4.4 External Factors

4.4.1 Influence of 

%noch etwas Fülltext
\blinddocument
