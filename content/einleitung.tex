% !TeX spellcheck = de_DE

\chapter{Introduction}

\section{Motivation}
The availability of accurate high resolution 3D information of roads with lane markings and road furniture plays an important role towards autonomous driving. Aerial imagery is a valuable database to derive 3D information of roads even in areas difficult to access, like on motorways. In view of the fact, that in Germany exists no area-wide 3D information of the road surface including lane markings, new methods to derive this information are demanded.

The standard workflow with aerial images would be to project the images onto a DSM and to derive the information in the projected imagery, but the generation of Digital Surface Model (DSM) from stereo images is challenging in the regions with low textures. The lane markings, for example, are the most visible texture on asphalt roads useful for 3D reconstruction. Thus, it is desired to improve the quality of the DSM on the road surfaces by exploiting the line character of the lane markings. 


%Inspired by the same concept as...
%At the moment an aerial image was taken, a bundle of rays travel from a lane marking on the ground to the sensor plate.



\section{3D Reconstruction}
detection + matching + reconstruction of the geometry at the exposure moment

\subsection{Dense Image Matching (DIM)}




\subsection{Semi-Global Matching (SGM)}

Such high-resolution DSM gives a good starting point for the lane marking refinement. Assuming that the final 3D correction is < 1m)


Digital Surface Models Generation by Semi-Global Matching
for semi-global matching, the matching step is cast into an energy minimization problem.
The smoothness is not strong
low texture area, not able to triangulate the 3D point, it took a min value as the height of that area -> systematic error



\subsection{Structure from Motion (SfM)}


\subsection{Scale-invariant Feature Transform(SIFT)}


\section{Feature Extraction and Matching}


\subsection{Issues in Line Matching}

Line matching is challenging for several reasons. First, line segments are often detected inexactly by automatic line detectors or occlusions appear in part of the lines. Thus the end-points of a line segment often do not correspond to each other in different views. Secondly, as mentioned above, there is no strong disambiguating geometric constraint available. Moreover, the corresponding neighborhoods may well have a very different shape and orientation, or even totally different surroundings when dealing with wiry objects. [C. Schmid et al 1997] [M. Hofer et al Feb. 2013]

C. Schmid et al. [C. Schmid et al. 1997] exploit the epipolar geometry of line segments and the one-parameter family of homographies to provide point-wise correspondences, allowing cross-correlation of patches around line segments along the candidate lines in the epipolar-beam-region for matching scores evaluation.

Considering the facts that the neighborhood of line segments may be of poorly textured or the surrounding area of a line segment on the intersection of two planes may have very different affine shape changes in different long base-line views, line segments are barely comparable using classical correlation patches yet the color neighborhood along this line segment undergoes only slight changes. Based on color histogram rather than textures, H. Bay et al. [H. Bay et al. 2005] exploit the appearance similarity of line segment pairs and their topological layout to iteratively increase the correct matches. If region matches are available, they are automatically integrated in the topological configuration and exploited in combination. The final coplanar grouping stage allows to estimate the fundamental matrix even from line segments only. While color provides a very strong cue for discrimination, it may fail in the case where color feature is not distinctive, e.g. gray images. Besides, the advantage of matching groups of line segments is that more geometric information is available for disambiguation, the disadvantage is the increased computational complexity [C. Schmid et al. 1997].

Without resorting to any other constraints or prior knowledge, Z. Wang et al. [Z. Wang et al. May. 2009] propose a purely image content-based line descriptor MSLD for automatic line segments matching. 

The above appearance-based approaches demand either constant and rich neighboring textures or similar color profile of line segments, they are technically matching the surroundings instead of the lines themselves.




\section{Purpose}

In this thesis, I develop a framework to automatically detect the lane markings in the unprojected aerial imagery, and to refine the 3D information of the road surface by exploiting the line character of the lane markings.

[solve the problem of (quasi)infinite and curved lane markings]
[apply standard line detection algorithms for automatic lane marking detection]
[use of aerial image data set with special flight configuration at both sides of the motorway]

The unprojected aerial images with their bundle-adjusted orientations and the DSM are the inputs of my research. By sliding a window of reasonable length and width through the curved long lane lines, the collected line segment in the current window is assumed to be straight. I investigate the use of linear regression to optimize the location of each line segments in object space so that its back projection would best fit the detected 2D line in all the covering views, i.e. the position and height of each 3D lane marking segments will be refined in one optimization step.

The framework will be tested on aerial imagery from the German highway A9 and validated with ground measured lane markings by GPS.


 dem Buch \cite{WSPA} zu tun.
\emph{Satz} eine neue Zeile im \texttt{.tex}-Dokument anzufangen.


\section*{Gliederung}
Die Arbeit ist in folgender Weise gegliedert:
\begin{description}
\item[Kapitel~\ref{chap:k2} -- \nameref{chap:k2}:] Hier werden werden die Grundlagen dieser Arbeit beschrieben.
\item[Kapitel~\ref{chap:zusfas} -- \nameref{chap:zusfas}] fasst die Ergebnisse der Arbeit zusammen und stellt Anknüpfungspunkte vor.
\end{description}
