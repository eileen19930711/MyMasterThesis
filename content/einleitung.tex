% !TeX spellcheck = de_DE

\chapter{Introduction}

\section{Motivation}
The availability of large-scale, accurate high resolution 3D information of roads with lane markings and road furniture plays an important role towards autonomous driving. Such redundant sources of information for supporting the execution of dynamic driving task, such as localization, environmental perception and behavioral planning, is useful for increasing the robustness and availability of the system. % (Fischer et al., 2017)
Aerial imagery is a valuable database to derive 3D information of roads even in areas difficult to access, like on motorways. Compared to optical satellite data, acquiring large-scale 3D lane markings by optical aerial imagery is more efficient and has higher accuracy and spatial resolution. In view of the fact, that in Germany exists no area-wide 3D information of the road surface including lane markings, new methods to derive this information are demanded.

The standard workflow with aerial images would be to project the images onto a DSM and to derive the information in the projected imagery, but the generation of Digital Surface Model (DSM) from stereo images is challenging in the regions with low textures. The lane markings, for example, are the most visible texture on asphalt roads useful for 3D reconstruction. Thus, it is desired to improve the quality of the DSM on the road surfaces by exploiting the line character of the lane markings. 




%%%%%%%%%%%%%%%%%%%%%%%%%%%%%%%%%%%%%%%
\section{3D Reconstruction}

Generally, the procedure of 3D objects reconstruction consists of feature extraction in image space and depth information recovery in object space. 
To reconstruct the depth information at the exposure moment, multi rays spatial intersection or single ray intersection with an elevation model can be applied.
With Multi View Stereo (MVS) images and known \gls{io} and \gls{eo} of different images, spatial intersection is usually applied in the cases that the correspondences of the extracted features among different views can be established.
Alternatively, when \gls{dem} being available, single extracted feature can be directly projected onto the DEM. In this case, the quality of DEM directly influences the result of 3D reconstruction. 

\subsection{Feature Extraction and Matching}
The aim of feature extraction is to gain the characteristics of the images, through which the stereo correspondence processes. As a result, the characteristics of the images closely link to the choice of matching methods. % https://en.wikipedia.org/wiki/3D_reconstruction
One of the appearance-based similarity measures for point (corner) features matching is NCC, which is the simplest but effective affine-illumination-invariant method. 

Harris corner detector rotation-invariant

Scale-Invariant Feature Transform (SIFT) is proposed by Lowe in 1999 [D.G. Lowe 1999]. Lowe‘s approach transforms an image into a large collection of local feature vectors, also known as keypoint descriptors. Each SIFT feature descriptor is invariant to image translation, scaling, and rotation, partially invariant to illumination changes and affine or 3D projection. These SIFT keypoints are than matched by identifying their nearest neighbors.


sfm, sgm, 


[However, it is challenging to apply line matching in practice. The reason is that extracted line features are often incomplete because of the defect of detectors or occlusions appearing in part of the lines. Moreover, the shapes, directions and surroundings of corresponding lane markings may be different in MVS, which makes line matching become failed (Schmid et al., 1997; Hofer et al., 2013).]. Line matching%的文獻回顧 [每個方法以何種做法解決了哪些問題,但還有那些問題沒解決]. 由於沒有一種文獻回顧是可以同時解決3種問題,因此
using space intersection to project lane markings to 3D object space%是不可行的。

%\subsection{Projection on DSM}




%%%%%%%%%%%%%%%%%%%%%%%%%%%%%%%%%%%%%%%
\section{\gls{dsm}}

High resolution DSM can be generated by laser scanning or dense image matching. Compared to laser scanning, applying dense image matching to produce DSM is of lower cost and of shorter time for data collection. Dense image matching is to derive pixel-wise corresponding image points in MVS, and Semi-Global Matching (SGM), firstly proposed by Hirschmuller in 2008, is one of the most popular algorithm for dense image matching. SGM considers the information of entire image by means of aggregating matching cost along 8 or 16 cost paths, which not only enhances matching quality but also reduces computational complexity. However, lane markings are often located on homogeneous road surfaces, which makes the result of SGM become unstable. Therefore, 3D lane marking reconstructed by single aerial image and DSM is unstable and contains lots of noise. %(放圖?)

Such high-resolution DSM gives a good starting point for the lane marking refinement.

%%%
for semi-global matching, the matching step is cast into an energy minimization problem.
The smoothness is not strong
low texture area, not able to triangulate the 3D point, it took a min value as the height of that area -> systematic error


% Dense Image Matching (DIM)
3D reconstruction using dense image matching is a hot topic as it enables the automatic extraction of 3D urban models, notably from airborne oblique imagery. However, applying DIM algorithms to oblique imagery is challenging because of large scale variations, illumination changes and the many occlusions.



\section{Issues in Line Matching}

Line matching is challenging for several reasons. First, line segments are often detected inexactly by automatic line detectors or occlusions appear in part of the lines. Thus the end-points of a line segment often do not correspond to each other in different views. Secondly, as mentioned above, there is no strong disambiguating geometric constraint available. Moreover, the corresponding neighborhoods may well have a very different shape and orientation, or even totally different surroundings when dealing with wiry objects. [C. Schmid et al 1997] 

\cite{HoferFeb2013}

As \citeauthor{HoferFeb2013} suggested in \citeyear{HoferFeb2013},...


\section{Related Work}

C. Schmid et al. [C. Schmid et al. 1997] exploit the epipolar geometry of line segments and the one-parameter family of homographies to provide point-wise correspondences, allowing cross-correlation of patches around line segments along the candidate lines in the epipolar-beam-region for matching scores evaluation.

Considering the facts that the neighborhood of line segments may be of poorly textured or the surrounding area of a line segment on the intersection of two planes may have very different affine shape changes in different long base-line views, line segments are barely comparable using classical correlation patches yet the color neighborhood along this line segment undergoes only slight changes. % Long sentence, maybe rewrite. This block is also too detailled and specific for an introduction. Maybe move to other chapter? 
Based on color histogram rather than textures, H. Bay et al. [H. Bay et al. 2005] exploit the appearance similarity of line segment pairs and their topological layout to iteratively increase the correct matches. If region matches are available, they are automatically integrated in the topological configuration and exploited in combination. The final coplanar grouping stage allows to estimate the fundamental matrix even from line segments only. While color provides a very strong cue for discrimination, it may fail in the case where color feature is not distinctive, e.g. gray images. Besides, the advantage of matching groups of line segments is that more geometric information is available for disambiguation, the disadvantage is the increased computational complexity [C. Schmid et al. 1997].


Without resorting to any other constraints or prior knowledge, Z. Wang et al. [Z. Wang et al. May. 2009] propose a purely image content-based line descriptor MSLD for automatic line segments matching. Adapting SIFT-like strategy, MSLD is highly distinctive and robust against image rotation, illumination change, image blur, viewpoint change noise, JPEG compression and partial occlusion [Z. Wang et al. May. 2009]

The above appearance-based approaches demand either constant and rich neighboring textures or similar color profile of line segments, they are technically matching the surroundings instead of the lines themselves.

1.1.3 3D line Reconstruction without Matching

In order to create 3D models without the need of explicit line matching, A. Jain et al. [A. Jain et al. 2010] generate all possible hypothetical straight 3D line segments by triangulating the detected straight 2D line segments from different views, then they keep the one whose back projection on the gradient images of neighboring views has the highest score, assuming that line segments correspond to high gradient areas in images. Built upon the same principles whilst applying epipolar constraint on line segment end-points, M. Hofer et al. generate less hypothetical 3D line segments and thus increase performance significantly while still creating accurate results. However, both approaches are barely possible in the case of infinite line reconstruction, where the detected 2D lines in different views do not exactly correspond to the same part of a  3D line.

Taylor et al. [C. J. Taylor et al. 1995] formulate the Structure from Motion (SfM) problem in terms of minimization of an objective function that measures the total squared distance in the image plane between the observed edge segments and the projections of the reconstructed lines. By reconstructing the infinite straight line that supports the observed edge segments rather than the end-points of the line, the algorithm can be used even when multiple edges in a single image correspond to different portions of the same 3D line.

\section{Purpose}

In this thesis, I develop a framework to automatically detect the lane markings in the unprojected aerial imagery, and to refine the 3D information of the road surface by exploiting the line character of the lane markings.

[solve the problem of (quasi)infinite and curved lane markings]
[apply standard line detection algorithms for automatic lane marking detection]
[use of aerial image data set with special flight configuration at both sides of the motorway]

The unprojected aerial images with their bundle-adjusted orientations and the DSM are the inputs of my research. By sliding a window of reasonable length and width through the curved long lane lines, the collected line segment in the current window is assumed to be straight. I investigate the use of linear regression to optimize the location of each line segments in object space so that its back projection would best fit the detected 2D line in all the covering views, i.e. the position and height of each 3D lane marking segments will be refined in one optimization step.

The framework will be tested on aerial imagery from the German highway A9 and validated with ground measured lane markings by GPS.
